%%%%%%%%%%%%%%%%%
% This is an sample CV template created using altacv.cls
% (v1.1.5, 1 December 2018) written by LianTze Lim (liantze@gmail.com). Now compiles with pdfLaTeX, XeLaTeX and LuaLaTeX.
%
%% It may be distributed and/or modified under the
%% conditions of the LaTeX Project Public License, either version 1.3
%% of this license or (at your option) any later version.
%% The latest version of this license is in
%%    http://www.latex-project.org/lppl.txt
%% and version 1.3 or later is part of all distributions of LaTeX
%% version 2003/12/01 or later.
%%%%%%%%%%%%%%%%

%% If you need to pass whatever options to xcolor
\PassOptionsToPackage{dvipsnames}{xcolor}

%% If you are using \orcid or academicons
%% icons, make sure you have the academicons
%% option here, and compile with XeLaTeX
%% or LuaLaTeX.
% \documentclass[10pt,a4paper,academicons]{altacv}

%% Use the "normalphoto" option if you want a normal photo instead of cropped to a circle
% \documentclass[10pt,a4paper,normalphoto]{altacv}

\documentclass[10pt,a4paper,ragged2e]{altacv}

%% AltaCV uses the fontawesome and academicon fonts
%% and packages.
%% See texdoc.net/pkg/fontawecome and http://texdoc.net/pkg/academicons for full list of symbols. You MUST compile with XeLaTeX or LuaLaTeX if you want to use academicons.

% Change the page layout if you need to
\geometry{left=1cm,right=9cm,marginparwidth=6.8cm,marginparsep=1.2cm,top=1.25cm,bottom=1.25cm}

% Change the font if you want to, depending on whether
% you're using pdflatex or xelatex/lualatex
\ifxetexorluatex{}
  % If using xelatex or lualatex:
  \setmainfont{Carlito}
\else
  % If using pdflatex:
  \usepackage[utf8]{inputenc}
  \usepackage[T1]{fontenc}
  \usepackage[default]{lato}
  \usepackage{import}
  \usepackage{pdfpages}
\fi

% Change the colours if you want to
\definecolor{Blue}{HTML}{0070C0}
\definecolor{SlateGrey}{HTML}{2B2C2E}
\definecolor{LightGrey}{HTML}{666666}
\colorlet{heading}{SlateGrey}
\colorlet{accent}{Blue}
\colorlet{emphasis}{SlateGrey}
\colorlet{body}{LightGrey}

\usepackage{hyperref}
\hypersetup{
  colorlinks=true
  urlcolor= Blue
}

% Change the bullets for itemize and rating marker
% for \cvskill if you want to
\renewcommand{\itemmarker}{{\small\textbullet}}
\renewcommand{\ratingmarker}{\faCircle}

%% sample.bib contains your publications
%\addbibresource{sample.bib}

\begin{document}
\name{Elias Joaquín Little}
\tagline{Full–Time Student}
\photo{2.8cm}{Headshot}
\personalinfo{%
  % Not all of these are required!
  % You can add your own with \printinfo{symbol}{detail}
  \email{EliasJLittle@gmail.com}
  \phone{512–788–4349}
  % \mailaddress{Address, Street, 00000 County}
  \location{Austin, Texas}
  \\
  \homepage{EliasL.com}
  % \twitter{@twitterhandle}
  \linkedin{linkedin.com/in/elias-little}
  \github{github.com/EliasLittle}
  %% You MUST add the academicons option to \documentclass, then compile with LuaLaTeX or XeLaTeX, if you want to use \orcid or other academicons commands.
  % \orcid{orcid.org/0000-0000-0000-0000}
}

%% Make the header extend all the way to the right, if you want.
\begin{fullwidth}
\makecvheader{}
\end{fullwidth}

%% Depending on your tastes, you may want to make fonts of itemize environments slightly smaller
% \AtBeginEnvironment{itemize}{\small}

%% Provide the file name containing the sidebar contents as an optional parameter to \cvsection.
%% You can always just use \marginpar{...} if you do
%% not need to align the top of the contents to any
%% \cvsection title in the "main" bar.
\cvsection[page1sidebar]{Projects}

\cvevent{Monte Carlo Credit}{}{Summer 2020}{}
Designed and built a service that analyzes someone's transactions and compares their spending habits against rewards from various credit cards to recommend the best one. Stack includes Python, Flask, Svelte, MondoDB, Plaid \& Stripe APIs(\href{https://montecarlocredit.com/}{Launching Soon})

\divider{}

\cvevent{Bright Future}{}{Summer 2020}{}
Built the backend for a site that scrapes well-known news sites with python and analyzes then presents the top articles based on positivity using the TextBlob natural language processing library. (\href{https://brightfuture.news}{Link})

\divider{}

\cvevent{Project Euler}{}{March 2017 - Present}{}
Solved 50+ Problems on \href{https://projecteuler.net}{ProjectEuler.net} (\href{https://projecteuler.net/profile/EliasLittle.png}{My progress})

%Optional Old Projects
% \cvevent{RSA Encryption}{LASA HS}{}{}
% Wrote RSA encryption and decryption algorithms implementing Pollard's Rho algorithm
%
% \divider{}
%
% \cvevent{Orchestra Database}{LASA HS}{}{}
% Cleaned existing spreadsheet library and converted it into a searchable SQL database

\cvsection{Experience}

\cvevent{Intern}{Applied Research Labs UT Austin}{Summer 2019}{Austin, TX}
\begin{itemize}
\item Designed, built, tested, and presented a prototype low-cost low-frequency underwater acoustic source.
\end{itemize}

\divider{}

\cvevent{Contractor}{Torzal Guitars}{Summer 2018}{Austin, TX}
\begin{itemize}
\item Researched, designed, and constructed a small-scale Enterprise Resource Planning database in Airtable
\end{itemize}

%Optional Old Work Experience
%\divider{}
%
%\cvevent{Luthier I}{Collings Guitars}{Summer 2017}{Austin, TX}
%\begin{itemize}
%\item Prepared incoming wood shipments for kiln \& acclimation; Sanded electric guitar bodies
%\end{itemize}

%\medskip

\cvsection{Education}

\cvevent{B.Eng.\ in Operations Research (ORIE)}{Cornell University}{Aug 2019 – Present}{Ithaca, NY}
\begin{itemize}
\item \textbf{GPA} – 3.65 | Engineering Ambassador \textbullet \, SHPE
\item \textbf{Courses}: Object Oriented Programming \& Data Structures, Discrete Structures, Linear Algebra, Multivariable Calculus, Introduction to Machine Learning, Applications of Operations Research, Introduction to Computational Thinking (unofficial, MIT 18.S191)
\end{itemize}

\divider{}

\cvevent{High School | \small\makebox[0.5\linewidth][l]{\faMagnet\hspace{0.5em}Magnet Endorsement}}{Liberal Arts \& Science Academy}{Aug 2015 – May 2019}{Austin, TX}
\begin{itemize}
\item \textbf{GPA} – 4.32 | SAT – 1560 \textbullet \, Orchestra President \textbullet \, Ambassador
\item \textbf{Courses}: AP Computer Science, Computational Problem Solving
\end{itemize}

\begin{comment}
\cvsection{A Day of My Life}
% Adapted from @Jake's answer from http://tex.stackexchange.com/a/82729/226
% \wheelchart{outer radius}{inner radius}{
% comma-separated list of value/text width/color/detail}
\wheelchart{1.5cm}{0.5cm}{%
  6/8em/accent!30/{Sleep,\\beautiful sleep},
  3/8em/accent!40/Hopeful novelist by night,
  8/8em/accent!60/Daytime job,
  2/10em/accent/Sports and relaxation,
  5/6em/accent!20/Spending time with family
}


\clearpage
\cvsection[page2sidebar]{Publications}

\nocite{*}

\printbibliography[heading=pubtype,title={\printinfo{\faBook}{Books}},type=book]

\divider

\printbibliography[heading=pubtype,title={\printinfo{\faFileTextO}{Journal Articles}},type=article]

\divider

\printbibliography[heading=pubtype,title={\printinfo{\faGroup}{Conference Proceedings}},type=inproceedings]

%% If the NEXT page doesn't start with a \cvsection but you'd
%% still like to add a sidebar, then use this command on THIS
%% page to add it. The optional argument lets you pull up the
%% sidebar a bit so that it looks aligned with the top of the
%% main column.
% \addnextpagesidebar[-1ex]{page3sidebar}
\end{comment}

\end{document}
