\documentclass[10pt,a4paper]{article}
%\usepackage[utf8]{inputenc}
%\usepackage[T1]{fontenc}
\usepackage[default]{lato}

\title{Essay Questions}
\date{\vspace{-5ex}}

\begin{document}
\maketitle
\begin{enumerate}
  \item How were you first introduced to Computer Science? How have you continued to develop your technical skills and seek additional exposure to the field? \par \smallskip
  I was first introduced to computer science by a friend of mine in high school. Since they first convinced me to learn javascript, I have taught myself multiple languages, worked on a few personal and school projects, and spent plenty of time working on various coding challenges such as those found on projectueler.net. Shortly after I started coding, I decided to take the AP Computer Science course at my high school, which taught me Java and a lot of basic principals of computer science, particularly object-oriented programming. From there my interest only grew stronger and I looked for more resources and opportunities. I have taken multiple online courses from various providers such as codecademy.com (where I just started their new Go course). My learning has been more or less been ad hoc as various questions or projects required me to learn something new, but I have also been learning in a more structured manner in my CS courses this semester.

  \item What is your strongest programming language? How much experience do you have using the language? Go into detail about how you used this technical language. If talking about a group project, be specific about your role in the final product. (Examples can include projects, coursework, competitions, websites, previous internships, etc.) \par \smallskip
  My strongest programming language is Python. I have only been programming in python for about a year, but because of the low learning curve, I was able to familiarize myself with the basics extremely quickly and then progress to learning more advanced techniques and other libraries and APIs. It is now my go-to language for most projects. One such project where I used python, was when I converted my high school orchestra’s spreadsheet of songs in our library, into a navigable SQL database. I downloaded the spreadsheet as a CSV then cleaned all of the data, once it was cleaned I used the MySQL connector library to connect to my local MySQL server and imported all of the cleaned data. This allowed the use of a database manager to browse our library more effectively. I also experimented using Tkinter to create a GUI for interacting with the database. Some other projects I’ve used python for include writing an RSA encryption and decryption program using Pollard’s Rho algorithm, and experimenting with the Alpaca Markets API for algorithmic stock trading. I am currently learning to better use Numpy, Pandas, MatPlotLib, and SKLearn in a machine learning class this semester.

  \newpage
  \item At Google, we believe that a diversity of perspectives, ideas, and cultures leads to the creation of better products and services. Tell us about your background and experiences and how they make you unique. \par \smallskip
  My father comes from a Quaker family and is a descendant of English and Welsh immigrants from the mid-18th century. My mother is a Zen Buddhist Mexican-American who grew up Catholic and a descendant of Mexicans living in what is now Texas since it was Mexico. My paternal great-grandfather won a Nobel peace prize and is the only Westerner to perform a Noh play for the Japanese Emperor, and my maternal grandparents were migrant farmworkers. All of these different factors have formed who I am. I am everpresent of the fact that everyone’s backgrounds are extremely different and affect everyone in different and significant ways. Cultural diversity brings unique viewpoints and perspectives to any conversation and is especially important when solving problems.

  \item List the technical courses you will be taking next semester, and please note which programming language(s) will be used, if applicable. If you have not registered for classes yet, please list the courses you plan on taking. \par \smallskip
  \begin{itemize}
    \item PHYS 1116: Honors Physics, Mechanics and Special Relativity
    \item MATH 2940: Linear Algebra
    \item MATH 2930: Differential Equations
    \item ENGRI 1101: Intro to Operations Research
    \item ENGRD 2700: Basic Engineering Probability and Statistics
  \end{itemize}
  Unfortunately, the next CS class that I am supposed to take does not align with my schedule for next semester. But I will still be working on personal projects and I will apply for the Data Science Project Team, which primarily uses python.

  \item List any clubs and/or organizations that you participate in.\par \smallskip
  \begin{itemize}
    \item Cornell Hyperloop Project Team - Propulsion Subteam
    \item Cornell ACSU
    \item Cornell SHPE
    \item Anything Goes Musical Troupe
    \item Cornell Open Orchestra
  \end{itemize}
\end{enumerate}
\nocite{*}
\end{document}
